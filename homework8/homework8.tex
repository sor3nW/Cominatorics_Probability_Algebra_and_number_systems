
\documentclass[12pt]{article}
\usepackage[margin=1in]{geometry} 
\usepackage{amsmath,amsthm,amssymb,scrextend, quoting}
\usepackage{fancyhdr}
\pagestyle{fancy}
\usepackage{enumitem}
\usepackage{graphicx}
 \usepackage{float}
\newcommand{\N}{\mathbb{N}}
\newcommand{\Z}{\mathbb{Z}}
\newcommand{\I}{\mathbb{I}}
\newcommand{\R}{\mathbb{R}}
\newcommand{\Q}{\mathbb{Q}}
\renewcommand{\qed}{\hfill$\blacksquare$}
\let\newproof\proof
\renewenvironment{proof}{\begin{addmargin}[1em]{0em}\begin{newproof}}{\end{newproof}\end{addmargin}\qed}

% \newcommand{\expl}[1]{\text{\hfill[#1]}$}
\newenvironment{theorem}[2][Theorem]{\begin{trivlist}
\item[\hskip \labelsep {\bfseries #1}\hskip \labelsep {\bfseries #2.}]}{\end{trivlist}}


\newenvironment{problem}[1]{%
  \begin{addmargin}[1em]{0em}
  \quoting
  \noindent
  \textbf{#1} % Display the problem number
  \normalfont % Ensure the text is in normal font
}{%
  \endquoting
  \end{addmargin}
  \qed
}


\newenvironment{sentence}{%
  \noindent
  \begin{quoting}[leftmargin=0em,rightmargin=2em]%
  \normalfont
}{%
  \end{quoting}%
}

 
\begin{document}

\lhead{MAT-2313-001}
\chead{Soren Watterson}
\rhead{April 12th, 2024}

\noindent\textbf{Problem 1} Find a generating function for the number of ways of organizing $k$ distinct photo files in four distinct folders assuming that:\\
\noindent\textbf{(a)} Each folder must be nonempty.\\
\begin{addmargin}[1em]{1em}
    Assume we have $4$ folders for holding our photos. We can represent these folders as
    \begin{align*}
        ()()()()
    \end{align*}
    Each folder can hold $k$ photos inside of it, which means that for each folder you can put in either $1$ photo, or $2$ photos, and so on. Therefore, the number of ways we can distribute the photos can be represented as
    \begin{align*}
        &(x + x^2 + x^3 + ... )^4
    \end{align*}
    Using polynomial expansion $(2)$ in Table $(6.1)$ we can then use this to create our \\generating function
    \begin{align*}
        \frac{1}{1-x} &= 1 + x + x^2 + ...\\
        x * \frac{1}{1-x} &= x + x^2 + x^3 + ...\\
        (x * \frac{1}{1-x})^4 &= (x + x^2 + x^3 + ...)^4\\
        (x * \frac{1}{1-x})^4 &= \frac{x^4}{(1-x)^4}\\
    \end{align*}
    In conclusion, our generating function is 
    \begin{align*}
        g(x) &= \frac{x^4}{(1-x)^4}
    \end{align*}
    \begin{flushright}
    \qed
    \end{flushright}
\end{addmargin}

\noindent\textbf{(b)} The first two folders accept only an even number of files.\\
\begin{addmargin}[1em]{1em}
    
    Assume again that we have our four folders expressed as
    \begin{align*}
        &()()()()
    \end{align*}
    However, the first two folder can now only accept an even number of photo files. This would mean that in the first two folders you can either put $0$ photo files, $2$ photo files, $4$ photo files, and so on as long as there is an even number of photo files in that folder. For the last two files we can put in $0$ files, $1$ file, $2$ files, and so on. This means that we can show the number of ways of distributing the photo files as
    \begin{align*}
        &(1 + x^2 + x^4 + ...)^2(1+x+x^2+x^3+...)^2\\
    \end{align*}

    We will then again use polynomial expansion $(2)$ to create our generating function

    \begin{align*}
        (\frac{1}{1-x})(\frac{1}{1-x}) &= (1 + x + x^2 + ...)(1 + x + x^2 + ...)\\
        (\frac{1}{1-x})^2(\frac{1}{1-x})^2 &= (1 + x + x^2 + ...)^2(1 + x + x^2 + ...)^2\\
        (\frac{1}{1-x^2})^2(\frac{1}{1-x})^2 &= (1 + x^2 + x^4 + ...)(1 + x + x^2 + ...)\\
        (\frac{1}{(1-x^2)^2})(\frac{1}{(1-x)^2})^2 &= (1 + x^2 + x^4 + ...)(1 + x + x^2 + ...)\\
    \end{align*}
    In conclusion, the generating function is 
    \begin{align*}
        g(x) &= (\frac{1}{(1-x^2)^2})(\frac{1}{(1-x)^2})^2\\
    \end{align*}
    \begin{flushright}
    \qed
    \end{flushright}
\end{addmargin}

\noindent\textbf{Problem 2} Repeat the preceding exercise (both parts) but assuming that within each folder, the pictures must appear in chronological order.\\

\noindent\textbf{2(a)} Each folder must be nonempty.\\
\begin{addmargin}[1em]{1em}
    In problem $1$ we were given the task of finding the generating function for the number of ways to distribute $k$ photo files among four distinct folders, but each folder must be nonempty. In that example the order of distributing those photo files did not matter. However, in this example we must find the generating function, but the order matters.\\\\
    First we can represent our four folders as 
    \begin{align*}
        &()()()()
    \end{align*}
    For each folder you must put at least $1$ photo file, but then each subsequent photo file has to be added to that box in chronological order. This means our folders can be shown as
    \begin{align*}
        (\frac{x}{1!} + \frac{x^2}{2!} + \frac{x^3}{3!} + ...)^4
    \end{align*}
    using example $3$ on page $273$ of Tucker we can find the generating function for this problem
    \begin{align*}
        (\frac{x}{1!} + \frac{x^2}{2!} + \frac{x^3}{3!} + ...)^4 &= (e^x-1)^4
    \end{align*}
    In conclusion our generating function for problem 2(a) is 
    \begin{align*}
        g(x) = (e^x-1)^4
    \end{align*}
    \begin{flushright}
    \qed
    \end{flushright}
\end{addmargin}


\noindent\textbf{2(b)} The first two folders accept only an even number of files. \\
\begin{addmargin}[1em]{1em}
    Just like the last example we have to find how to distribute $k$ photo files among four distinct folders, however, the photos in each folder have to show up in chronological order, and the first two folders can only accept an even number of photo files.\\\\
    We can represent the four folders as
    \begin{align*}
        ()()()()
    \end{align*}
    Inside the first two folders you can put $0$ files, $2$ files, $4$ files, or so on as long as they are in chronological order. Then in the other two folders you can put $1$ file, $2$ files, or so on as long as they are also in chronological order. Therefore we can show our folders as 
    \begin{align*}
        (1 + \frac{x^2}{2!} + \frac{x^4}{4!} + ...)^2
        (1+ \frac{x}{1!} + \frac{x^2}{2!} + \frac{x^3}{3!} + ...)^2
    \end{align*}
    using example $3$ from page $273$ of Tucker we can then create the generating function for this problem
    \begin{align*}
        \frac{1}{2}(e^x + e^{-x}) &= 1 + \frac{x^2}{2!} + \frac{x^4}{4!} + ...\\
        (\frac{1}{2}(e^x + e^{-x}))^2 (e^x)^2 &= (1 + \frac{x^2}{2!} + \frac{x^4}{4!} + ...)^2(1 + \frac{x^2}{2!} + \frac{x^4}{4!} + ...)^2\\
        (\frac{1}{2}(e^x + e^{-x}))^2 (e^x)^2 &= (\frac{1}{2}(e^x+e^{-x}))^2e^{2x}
    \end{align*}
    Therefore, our generating function is 
    \begin{align*}
        g(x) = (\frac{1}{2}(e^x+e^{-x}))^2e^{2x}
    \end{align*}
    \begin{flushright}
    \qed
    \end{flushright}
\end{addmargin}

\noindent\textbf{Problem 3} Find and solve a recurrence equation for the number of ways to park red, blue and black shipping containers on a row $n$ parking space row if the blue and black containers take only one parking space, but the red containers are double and take up two spaces. \\

\begin{addmargin}[1em]{1em}
    First to find the recurrence equation for the number of ways to park the shipping containers we can consider a basic case of the problem. Lets assume that in our example case there are two parking spots in the row. This means that the following combinations of parking can occur.\\
    \begin{figure}[H] 
        \centering
        \includegraphics[width=0.5\linewidth]{combinatorics and probability 2.png}
        \label{fig:enter-label}
    \end{figure}
    \noindent With two parking spaces there is a total of $5$ different ways to park the shipping containers. Consider instead a case where there are $n$ different parking spots. There are three types of ways that you can fill $n$ spots. First, we can fill all the spots up to spot $n-1$, then put a blue shipping container.
    \begin{figure}[H]
        \centering
        \includegraphics[width=0.5\linewidth]{combinatorics and probability 5.png}
        \label{fig:enter-label}
    \end{figure}
    \noindent Second, we can fill all the spots up to spot $n-1$, then put a black shipping container.
    \begin{figure}[H]
        \centering
        \includegraphics[width=0.5\linewidth]{combinatorics and probability 4.png}
        \label{fig:enter-label}
    \end{figure}
    \noindent Finally, we can fill all of the spots up to spot $n-2$, then put a red shipping container.
    \begin{figure}[H]
        \centering
        \includegraphics[width=0.5\linewidth]{combinatorics and probability 3.png}
        \label{fig:enter-label}
    \end{figure}
    \noindent  Therefore, the recurrence equation for the number of ways to park the red, blue, and black shipping containers is 
    \begin{align*}
        a_n = 2a_{n-1} + a_{n-2}
    \end{align*}
    Next we have to solve the recurrence equation.
    \begin{align*}
        a_n &= 2a_{n-1} + a_{n-2}\\
        a^n &= 2a^{n-1} + a^{n-1}\\
        a^2 &= 2a + 1\\
        0 &= a^2 -2a -1
    \end{align*}
    Now that we have our quadratic equation we can use the quadratic formula to find the solutions to our equation.
    \begin{align*}
        &\frac{-(-2)\pm \sqrt{(-2)^2-4(1)(-1)}}{2(1)}\\
        &= \frac{2 \pm \sqrt{8}}{2}\\
        &= \frac{2 \pm 2\sqrt{2}}{2}\\
        &= 1 \pm \sqrt{2}\\
        x &= 1 + \sqrt{2}, x = 1 - \sqrt{2}
    \end{align*}
    These roots give us the general solution
    \begin{align*}
        a_n &= A_1(1 + \sqrt{2})^n + A_2(1 - \sqrt{2})^n
    \end{align*}
    And we can now solve for $A_1$ and $A_2$
    \begin{align*}
        a_n &= A_1(1 + \sqrt{2})^n + A_2(1 - \sqrt{2})^n\\
        a_0 &= 1 = A_1(1 + \sqrt{2})^0 + A_2(1 - \sqrt{2})^0\\
        1 &= A_1 + A_2\\
        A_1 &= 1 - A_2 \\
        a_1 &= 2 = A_1(1 + \sqrt{2})^1 + A_2(1 - \sqrt{2})^1\\
        2 &= A_1(1 + \sqrt{2}) + A_2(1 - \sqrt{2})\\
        2 &= (1 - A_2)(1 + \sqrt{2}) + A_2(1 - \sqrt{2})\\
        2 &= 1 + \sqrt{2} - A_2 - A_2\sqrt{2} + A_2 - A_2\sqrt{2}\\
        2 &= 1 + \sqrt{2} - 2\sqrt{2}A_2\\
        A_2 &= \frac{1-\sqrt{2}}{-2\sqrt{2}}\\
    \end{align*}
    now we also solve for $A_1$
    \begin{align*}
        a_0 &= 1 = A_1(1 + \sqrt{2})^0 + (\frac{1-\sqrt{2}}{-2\sqrt{2}})(1-\sqrt{2})^0\\
        1 &= A_1 + \frac{1-\sqrt{2}}{-2\sqrt{2}}\\
        A_1 &= 1 - \frac{1-\sqrt{2}}{-2\sqrt{2}}
    \end{align*}
    Therefore, the solution to our recurrence relation is 
    \begin{align*}
        a_n &= (1 - \frac{1-\sqrt{2}}{-2\sqrt{2}})(1 + \sqrt{2})^n + (\frac{1-\sqrt{2}}{-2\sqrt{2}})(1-\sqrt{2})^n\\
    \end{align*}
    \begin{flushright}
    \qed
    \end{flushright}
\end{addmargin}


\noindent\textbf{Problem 4} Find and solve a recurrence equation for the number of words of length $n$ formed with the letters $A$, $B$, $C$ that do not contain the string "$CA$"\\
\begin{addmargin}[1em]{1em}
    The task of this problem is to find $a_n$, the number of $n$-digit sequences using the letters $A, B$ and $C$, but not containing the sequence $CA$.This problem is similar to Example $7$: A Forbidden Sequence on page $288$ of Tucker. If the first letter in our sequence is $A$ or $B$ then there are $a_{n-1}(n-1)$-length sequences. However, if the first letter in our sequence is $C$, then there are $a_{n-1}-a_{n-2}(n-1)$-length sequences that do not contain the sequence $CA$. Therefore our generating function is
    \begin{align*}
        a_n &= 3a_{n-1}-a_{n-2}
    \end{align*}
    Now we can solve our recurrence equation. 
    \begin{align*}
        a^n &=  3a^{n-1}-a^{n-2}\\
        a^2 &= 3a - 1\\
        0 &= a^2 -3a + 1
    \end{align*}
    Using this quadratic equation and the quadratic formula we can find the solutions to this equation that we will use in the general solution of our recurrence relation.
    \begin{align*}
        &\frac{3 \pm \sqrt{(-3)^2-4(1)(1)}}{2}\\
        &=\frac{3 \pm \sqrt{9-4(1)(1)}}{2}\\
        &=\frac{3 \pm \sqrt{5}}{2}\\
        x &= \frac{3 + \sqrt{5}}{2}, x = \frac{3 - \sqrt{5}}{2}\\
    \end{align*}
    With these solutions we now have our general solution
    \begin{align*}
        a_n &= A_1(\frac{3 + \sqrt{5}}{2})^n + A_2(\frac{3 - \sqrt{5}}{2})^n\\
        a_0 &= 1 = A_1(\frac{3 + \sqrt{5}}{2})^0 + A_2(\frac{3 - \sqrt{5}}{2})^0\\
        1 &= A_1 + A_2\\
        A_1 &= 1 - A_2\\
        a_1 &= 3 = A_1(\frac{3 + \sqrt{5}}{2})^1 + A_2(\frac{3 - \sqrt{5}}{2})^1\\
        3 &= (1- A_2)(\frac{3 + \sqrt{5}}{2}) + A_2(\frac{3 - \sqrt{5}}{2})\\
        3 &= (\frac{3 + \sqrt{5}}{2}) - \frac{A_2(3 + \sqrt{5})}{2} + A_2(\frac{3 - \sqrt{5}}{2})\\
        3 &= \frac{3 + \sqrt{5} - 3A_2 - \sqrt{5}A_2 + 3A_2 - \sqrt{5}A_2}{2}\\
        3 &= \frac{3 + \sqrt{5} - 2\sqrt{5}A_2}{2}\\
        A_2 &= \frac{3-\sqrt{5}}{-2\sqrt{5}}\\
    \end{align*}
    Now we can substitute this into the equation to solve for $A_1$\\
    \begin{align*}
        a_0 &= 1 = A_1 + \frac{3-\sqrt{5}}{-2\sqrt{5}}\\
        A_1 &= 1 - \frac{3-\sqrt{5}}{-2\sqrt{5}}
    \end{align*}
    Therefore, we have the solution to our recurrence relation
    \begin{align*}
        a_n &= (1 - \frac{3-\sqrt{5}}{-2\sqrt{5}})(\frac{3 + \sqrt{5}}{2})^n + (\frac{3-\sqrt{5}}{-2\sqrt{5}})(\frac{3 - \sqrt{5}}{2})^n\\
    \end{align*}
    \begin{flushright}
    \qed
    \end{flushright}
\end{addmargin}
\end{document}
